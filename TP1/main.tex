\documentclass[11pt]{article}
\usepackage{theme}
\usepackage{shortcuts}
% Document parameters
% Document title
\title{Assignment 1 (ML for TS) - MVA}
\author{
Fotios Kapotos \email{fotiskapotos@gmail.com} \\ % student 1
Adonis Jamal \email{adonis.jamal@student-cs.fr} % student 2
}

\begin{document}
\maketitle

\section{Introduction}

\paragraph{Objective.} This assignment has three parts: questions about convolutional dictionary learning, spectral features, and a data study using the DTW. 

\paragraph{Warning and advice.} 
\begin{itemize}
    \item Use code from the tutorials as well as from other sources. Do not code yourself well-known procedures (e.g., cross-validation or k-means); use an existing implementation. 
    \item The associated notebook contains some hints and several helper functions.
    \item Be concise. Answers are not expected to be longer than a few sentences (omitting calculations).
\end{itemize}



\paragraph{Instructions.}
\begin{itemize}
    \item Fill in your names and emails at the top of the document.
    \item Hand in your report (one per pair of students) by Sunday 9\textsuperscript{th} November 23:59 PM.
    \item Rename your report and notebook as follows:\\ \texttt{FirstnameLastname1\_FirstnameLastname2.pdf} and\\ \texttt{FirstnameLastname1\_FirstnameLastname2.ipynb}.\\
    For instance, \texttt{LaurentOudre\_ValerioGuerrini.pdf}.
    \item Upload your report (PDF file) and notebook (IPYNB file) using this link: \footnotesize{LINK}.
\end{itemize}


\section{Convolution dictionary learning}

\begin{exercise}
Consider the following Lasso regression:
\begin{equation}\label{eq:lasso}
    \min_{\beta\in\RR^p} \frac{1}{2}\norm{y-X\beta}^2_2 \quad + \quad \lambda \norm{\beta}_1
\end{equation}
where $y\in\RR^n$ is the response vector, $X\in\RR^{n\times p}$ the design matrix, $\beta\in\RR^p$ the vector of regressors and $\lambda>0$ the smoothing parameter.

Show that there exists $\lambda_{\max}$ such that the minimizer of~\eqref{eq:lasso} is $\mathbf{0}_p$ (a $p$-dimensional vector of zeros) for any $\lambda > \lambda_{\max}$. 
\end{exercise}

\begin{solution}  % ANSWER HERE
    We denote by \( F_\lambda(\beta) \) the Lasso objective.  
    For every \( \beta \),
    \begin{equation*}
    \begin{aligned}
    F_\lambda(\beta) - F_\lambda(0)
    &= \frac{1}{2}\|y - X\beta\|_2^2 + \lambda \|\beta\|_1 - \frac{1}{2}\|y\|_2^2 \\[6pt]
    &= -\,y^\top X\beta + \frac{1}{2}\|X\beta\|_2^2 + \lambda \|\beta\|_1.
    \end{aligned}
    \end{equation*}

    Or, using Hölder's inequality,
    \begin{equation*}
        |y^\top X\beta| \;\leq\; \|X^\top y\|_\infty \, \|\beta\|_1.
    \end{equation*}

    Hence,
    \begin{equation*}
        F_\lambda(\beta) - F_\lambda(0)
        \;\geq\;
        \left( \lambda - \|X^\top y\|_\infty \right) \|\beta\|_1
        + \frac{1}{2}\|X\beta\|_2^2.
    \end{equation*}

By noting that \(\lambda_{\max} = \|X^\top y\|_\infty\), we can conclude that for all \(\lambda \geq \lambda_{\max}\), the unique minimizer is \(\beta = 0\).

\end{solution}

\begin{exercise}
For a univariate signal $\mathbf{x}\in\mathbb{R}^n$ with $n$ samples, the convolutional dictionary learning task amounts to solving the following optimization problem:

\begin{equation}
\min_{(\mathbf{d}_k)_k, (\mathbf{z}_k)_k \\ \norm{\mathbf{d}_k}_2^2\leq 1} \quad\norm{\mathbf{x} - \sum_{k=1}^K \mathbf{z}_k * \mathbf{d}_k }^2_2 \quad + \quad\lambda \sum_{k=1}^K \norm{\mathbf{z}_k}_1
\end{equation}

where $\mathbf{d}_k\in\mathbb{R}^L$ are the $K$ dictionary atoms (patterns), $\mathbf{z}_k\in\mathbb{R}^{N-L+1}$ are activations signals, and $\lambda>0$ is the smoothing parameter.

Show that
\begin{itemize}
    \item for a fixed dictionary, the sparse coding problem is a lasso regression (explicit the response vector and the design matrix);
    \item for a fixed dictionary, there exists $\lambda_{\max}$ (which depends on the dictionary) such that the sparse codes are only 0 for any $\lambda > \lambda_{\max}$. 
\end{itemize}
\end{exercise}

\begin{solution}  % ANSWER HERE

For a fixed dictionnary the problem becomes

\begin{equation*}
\min_{(\mathbf{z}_k)_k \\ \norm{\mathbf{d}_k}_2^2\leq 1} \quad\norm{\mathbf{x} - \sum_{k=1}^K \mathbf{z}_k * \mathbf{d}_k }^2_2 \quad + \quad\lambda \sum_{k=1}^K \norm{\mathbf{z}_k}_1
\end{equation*}

Since convolution with a fixed atom $\mathbf{d}_k$ is linear in $\mathbf{z}_k$, there exists a matrix 
$D_k$ such that $D_k \mathbf{z}_k = \mathbf{z}_k * \mathbf{d}_k$.
Stacking all activations $\mathbf{z}_k$ into 
$\boldsymbol{\beta} = [\mathbf{z}_1^\top,\dots,\mathbf{z}_K^\top]^\top$
and defining 
$X = [D_1 \; D_2 \; \cdots \; D_K]$, the problem becomes
\[
\min_{\boldsymbol{\beta}} 
\|\mathbf{x} - X\boldsymbol{\beta}\|_2^2 + \lambda \|\boldsymbol{\beta}\|_1,
\]
which is exactly a Lasso regression with response $\mathbf{x}$ and design matrix $X$.

To answer the following question we just apply the result proved in the first question. 



\end{solution}

\section{Spectral feature}

Let $X_n$ ($n=0,\dots, N-1)$ be a weakly stationary random process with zero mean and autocovariance function $\gamma(\tau):= \mathbb{E}(X_n X_{n+\tau})$.
Assume the autocovariances are absolutely summable, \ie $\sum_{\tau\in\mathbb{Z}} |\gamma(\tau)| < \infty$, and square summable, \ie $\sum_{\tau\in\mathbb{Z}} \gamma^2(\tau) < \infty$.
Denote the sampling frequency by $f_s$, meaning that the index $n$ corresponds to the time $n / f_s$. For simplicity, let $N$ be even.


The \textit{power spectrum} $S$ of the stationary random process $X$ is defined as the Fourier transform of the autocovariance function:
\begin{equation}
    S(f) := \sum_{\tau=-\infty}^{+\infty}\gamma(\tau)e^{-2\pi f\tau/f_s}.
\end{equation}
The power spectrum describes the distribution of power in the frequency space. 
Intuitively, large values of $S(f)$ indicate that the signal contains a sine wave at the frequency $f$.
There are many estimation procedures to determine this important quantity, which can then be used in a machine-learning pipeline.
In the following, we discuss the large sample properties of simple estimation procedures and the relationship between the power spectrum and the autocorrelation.

(Hint: use the many results on quadratic forms of Gaussian random variables to limit the number of calculations.)

\begin{exercise}
In this question, let $X_n$ ($n=0,\dots,N-1)$ be a Gaussian white noise.

\begin{itemize}
    \item Calculate the associated autocovariance function and power spectrum. (By analogy with the light, this process is called ``white'' because of the particular form of its power spectrum.)
\end{itemize}

\end{exercise}

\begin{solution}
By definition of (zero-mean) Gaussian white noise, the samples $(X_n)$ are i.i.d. with
$\mathbb{E}[X_n] = 0$ and $\mathrm{Var}(X_n)=\sigma^2$.

The autocovariance function is
\[
\gamma_X[k] \;=\; \mathbb{E}[X_n X_{n+k}]
\;=\;
\begin{cases}
\sigma^2, & k = 0,\\
0, & k \neq 0~,
\end{cases}
\]
i.e. $\gamma_X[k] = \sigma^2 \,\delta[k]$.

The power spectrum (power spectral density), defined as the discrete-time Fourier transform of $\gamma_X[k]$, is then
\[
S_X(\omega)
= \sum_{k=-\infty}^{\infty} \gamma_X[k] e^{-j\omega k}
= \sigma^2.
\]

Therefore, the spectrum is flat (constant in $\omega$), which motivates the term ``white''.
\end{solution}



\begin{exercise}
A natural estimator for the autocorrelation function is the sample autocovariance
\begin{equation}
    \hat{\gamma}(\tau) := (1/N) \sum_{n=0}^{N-\tau-1} X_n X_{n+\tau}
\end{equation}
for $\tau=0,1,\dots,N-1$ and $\hat{\gamma}(\tau):=\hat{\gamma}(-\tau)$ for $\tau=-(N-1),\dots,-1$.
\begin{itemize}
    \item Show that $\hat{\gamma}(\tau)$ is a biased estimator of $\gamma(\tau)$ but asymptotically unbiased.
    What would be a simple way to de-bias this estimator?
\end{itemize}

\end{exercise}

\begin{solution}
\begin{equation*}
    \begin{split}
   \mathbb{E}[\hat{\gamma}(\tau)] & = (1/N) \sum_{n=0}^{N-\tau-1} \mathbb{E}[X_n X_{n+\tau}] \\
                                  & =  \frac{N-\tau}{N}\gamma(\tau)
    \end{split}
\end{equation*}

Thus $\hat{\gamma}(\tau)$ is biased for any finite $N$ unless $\tau=0$. 

However,
\[
\lim_{N \to \infty} \mathbb{E}[\hat{\gamma}(\tau)]
= \lim_{N \to \infty} \frac{N-\tau}{N} \gamma(\tau)
= \gamma(\tau),
\]
so the estimator is asymptotically unbiased.

We can debias this estimator by replacing the division by $N$ by $N-\tau$
    

\end{solution}

\begin{exercise}
Define the discrete Fourier transform of the random process $\{X_n\}_n$ by
\begin{equation}
    J(f) := (1/\sqrt{N})\sum_{n=0}^{N-1} X_n e^{-2\pi\iu f n/f_s}
\end{equation}
The \textit{periodogram} is the collection of values $|J(f_0)|^2$, $|J(f_1)|^2$, \dots, $|J(f_{N/2})|^2$ where $f_k = f_s k/N$.
(They can be efficiently computed using the Fast Fourier Transform.)
\begin{itemize}
    \item Write $|J(f_k)|^2$ as a function of the sample autocovariances.
    \item For a frequency $f$, define $f^{(N)}$ the closest Fourier frequency $f_k$ to $f$.
    Show that $|J(f^{(N)})|^2$ is an asymptotically unbiased estimator of $S(f)$ for $f>0$.
\end{itemize}
\end{exercise}

\begin{solution}
We have
\begin{equation*}
|J(f_k)|^2 
= \frac{1}{N}\sum_{n=0}^{N-1}\sum_{m=0}^{N-1}
X_n X_m\, e^{-2\pi i (k/N)(n-m)}.
\end{equation*}
Grouping terms with the same lag $\tau = n-m$ yields
\begin{equation*}
|J(f_k)|^2 
= \sum_{\tau=-(N-1)}^{N-1} 
\hat{\gamma}(\tau)\, e^{-2\pi i (k/N)\tau},
\end{equation*}
where 
\begin{equation*}
\hat{\gamma}(\tau)
= \frac{1}{N}\sum_{n=0}^{N-1-|\tau|} X_n X_{n+|\tau|},
\qquad
\hat{\gamma}(-\tau)=\hat{\gamma}(\tau).
\end{equation*}
Hence, the periodogram is the discrete Fourier transform of the sample autocovariance sequence.

Recall that the power spectral density is
\begin{equation*}
S(f) \;=\; \sum_{\tau=-\infty}^{\infty} \gamma(\tau)\,
e^{-2\pi i f \tau / f_s},
\end{equation*}

From the first part,
\begin{equation*}
|J(f_k)|^2 
= \sum_{\tau=-(N-1)}^{N-1} 
\hat{\gamma}(\tau)\, e^{-2\pi i (k/N)\tau},
\qquad
f_k = \frac{f_s k}{N}.
\end{equation*}

Taking expectation,
\begin{equation*}
\mathbb{E}\big[ |J(f_k)|^2 \big]
= \sum_{\tau=-(N-1)}^{N-1} 
\mathbb{E}[\hat{\gamma}(\tau)]\, e^{-2\pi i (k/N)\tau}.
\end{equation*}

From the previous question, $\mathbb{E}[\hat{\gamma}(\tau)]
= \frac{N-|\tau|}{N}\,\gamma(\tau)$. (The absolute value is used here to take negative $\tau$ values)
Thus,
\begin{equation*}
\mathbb{E}\big[ |J(f_k)|^2 \big]
= \sum_{\tau=-(N-1)}^{N-1} 
\frac{N-|\tau|}{N}\,\gamma(\tau)\,
e^{-2\pi i (k/N)\tau}.
\end{equation*}

As $N\to\infty$, we have $\frac{N-|\tau|}{N}\to 1$ for each fixed $\tau$, and
the finite sum converges to the bilateral sum defining $S(f)$ evaluated at 
$f = f_k = f_s k/N$. Hence
\begin{equation*}
\lim_{N\to\infty} 
\mathbb{E}\big[ |J(f_k)|^2 \big]
= S(f_k).
\end{equation*}

Finally, for an arbitrary $f>0$, let $f^{(N)}$ be the closest Fourier frequency
$f_k$. Then $f^{(N)} \to f$ as $N\to\infty$, and by continuity of $S(f)$,
\begin{equation*}
\lim_{N\to\infty} 
\mathbb{E}\big[ |J(f^{(N)})|^2 \big]
= S(f).
\end{equation*}
Therefore $|J(f^{(N)})|^2$ is an asymptotically unbiased estimator of $S(f)$
for $f>0$.

\end{solution}


\begin{exercise}\label{ex:wn-exp}
    In this question, let $X_n$ ($n=0,\dots,N-1)$ be a Gaussian white noise with variance $\sigma^2=1$ and set the sampling frequency to $f_s=1$ Hz
    \begin{itemize}
        \item For $N\in\{200, 500, 1000\}$, compute the \textit{sample autocovariances} ($\hat{\gamma}(\tau)$ vs $\tau$) for 100 simulations of $X$.
        Plot the average value as well as the average $\pm$, the standard deviation.
        What do you observe?
        \item For $N\in\{200, 500, 1000\}$, compute the \textit{periodogram} ($|J(f_k)|^2$ vs $f_k$) for 100 simulations of $X$.
        Plot the average value as well as the average $\pm$, the standard deviation.
        What do you observe?
    \end{itemize}
    Add your plots to Figure~\ref{fig:wn-exp}.
    
\begin{figure}
    \centering
    \begin{minipage}[t]{0.3\textwidth}
    \centerline{\includegraphics[width=\textwidth]{example-image-golden}}
    \centerline{Autocovariance ($N=200$)}
    \end{minipage}
    \begin{minipage}[t]{0.3\textwidth}
    \centerline{\includegraphics[width=\textwidth]{example-image-golden}}
    \centerline{Autocovariance ($N=500$)}
    \end{minipage}
    \begin{minipage}[t]{0.3\textwidth}
    \centerline{\includegraphics[width=\textwidth]{example-image-golden}}
    \centerline{Autocovariance ($N=1000$)}
    \end{minipage}
    \vskip1em
    \begin{minipage}[t]{0.3\textwidth}
    \centerline{\includegraphics[width=\textwidth]{example-image-golden}}
    \centerline{Periodogram ($N=200$)}
    \end{minipage}
    \begin{minipage}[t]{0.3\textwidth}
    \centerline{\includegraphics[width=\textwidth]{example-image-golden}}
    \centerline{Periodogram ($N=500$)}
    \end{minipage}
    \begin{minipage}[t]{0.3\textwidth}
    \centerline{\includegraphics[width=\textwidth]{example-image-golden}}
    \centerline{Periodogram ($N=1000$)}
    \end{minipage}
    \vskip1em
    \caption{Autocovariances and periodograms of a Gaussian white noise (see Question~\ref{ex:wn-exp}).}
    \label{fig:wn-exp}
\end{figure}

\end{exercise}

\begin{solution}
    
\end{solution}


%%% QUESTION 7 %%%
\begin{exercise}
    We want to show that the estimator $\hat{\gamma}(\tau)$ is consistent, \ie it converges in probability when the number $N$ of samples grows to $\infty$ to the true value ${\gamma}(\tau)$.
    In this question, assume that $X$ is a wide-sense stationary \textit{Gaussian} process.
    \begin{itemize}
        \item Show that for $\tau>0$ 
    \begin{equation}
       \text{var}(\hat{\gamma}(\tau)) = (1/N) \sum_{n=-(N-\tau-1)}^{n=N-\tau-1} \left(1 - \frac{\tau + |n|}{N}\right) \left[\gamma^2(n) + \gamma(n-\tau)\gamma(n+\tau)\right].
    \end{equation}
    (Hint: if $\{Y_1, Y_2, Y_3, Y_4\}$ are four centered jointly Gaussian variables, then $\mathbb{E}[Y_1 Y_2 Y_3 Y_4] = \mathbb{E}[Y_1 Y_2]\mathbb{E}[Y_3 Y_4] + \mathbb{E}[Y_1 Y_3]\mathbb{E}[Y_2 Y_4] + \mathbb{E}[Y_1 Y_4]\mathbb{E}[Y_2 Y_3]$.) 
    \item Conclude that $\hat{\gamma}(\tau)$ is consistent.
    \end{itemize}
\end{exercise}

\begin{solution}
    Let $\tau > 0$. We compute $\mathbb{E}[\hat{\gamma}^2(\tau)]$:\\
    $\begin{aligned}
        \mathbb{E}[\hat{\gamma}^2(\tau)]
        &= \mathbb{E} \left[ \frac{1}{N^2} \sum_{0\leq n, m \leq N - \tau -1} X_n X_{n + \tau} X_m X_{m + \tau} \right]\\
        &= \frac{1}{N^2} \sum_{0\leq n, m \leq N - \tau -1} \mathbb{E}[X_n X_{n + \tau} X_m X_{m + \tau}]
    \end{aligned}$\\
    Using the hint, since we assume that $X$ is a Gaussian process, we have:\\
    $\begin{aligned}
        \mathbb{E}[\hat{\gamma}^2(\tau)]
        &= \frac{1}{N^2} \sum_{0\leq n, m \leq N - \tau -1} \mathbb{E}[X_n X_{n + \tau}] \mathbb{E}[X_m X_{m + \tau}] + \mathbb{E}[X_n X_m] \mathbb{E}[X_{n + \tau} X_{m + \tau}] + \mathbb{E}[X_n X_{m + \tau}] \mathbb{E}[X_m X_{n + \tau}]\\
        &= \frac{1}{N^2} \sum_{0\leq n, m \leq N - \tau -1} \gamma^2(\tau) + \gamma^2(n - m) + \gamma(n - m + \tau) \gamma(n - m - \tau)\\
        &= \frac{(N - \tau)^2}{N^2} \gamma^2 (\tau) + \frac{1}{N^2} \sum_{0\leq n, m \leq N - \tau -1} \gamma^2(n - m) + \gamma(n - m + \tau) \gamma(n - m - \tau)
    \end{aligned}$\\
    We know that $\var (\hat{\gamma}(\tau)) = \mathbb{E}[\hat{\gamma}^2(\tau)] - (\mathbb{E}[\hat{\gamma}(\tau)])^2$. From the previous exercise, we have $\mathbb{E}[\hat{\gamma}(\tau)] = \frac{N - \tau}{N} \gamma(\tau)$. Therefore, $$(\mathbb{E}[\hat{\gamma}(\tau)])^2 = \frac{(N - \tau)^2}{N^2} \gamma^2 (\tau)$$
    By noticing that this term simplifies with the previous result, we can now express the variance as:
    $$\var (\hat{\gamma}(\tau)) = \frac{1}{N^2} \sum_{0\leq n, m \leq N - \tau -1} \gamma^2(n - m) + \gamma(n - m + \tau) \gamma(n - m - \tau)$$
    The double summation only depends on the difference $k = n - m$. By changing the variables, we have for a given $k$, $(N - \tau - |k|)$ pairs $(n, m)$. We get:\\
    $\begin{aligned}
        \var (\hat{\gamma}(\tau)) 
        &= \frac{1}{N^2} \sum_{n = -(N - \tau - 1)}^{N - \tau - 1} (N - \tau - |n|) [\gamma^2 (n) + \gamma (n - \tau) \gamma (n + \tau)]\\
        &= \frac{1}{N} \sum_{n = -(N - \tau - 1)}^{N - \tau - 1} \frac{N - \tau - |n|}{N} [\gamma^2 (n) + \gamma (n - \tau) \gamma (n + \tau)]\\
        &= \frac{1}{N} \sum_{n = -(N - \tau - 1)}^{N - \tau - 1}  \left( 1 - \frac{\tau + |n|}{N}\right) [\gamma^2 (n) + \gamma (n - \tau) \gamma (n + \tau)]
    \end{aligned}$\\
    The obtained result is exactly what we were looking for.

    From exercise 4, we know that $\lim_{N \to \infty} \mathbb{E}[\hat{\gamma}(\tau)] = \gamma(\tau)$, so the estimator is asymptotically unbiased.

    We need to show that $\lim_{N \to \infty} \var(\hat{\gamma}(\tau)) = 0$ to prove that $\hat{\gamma}(\tau)$ is consistent. 
    
    We bound $\var (\hat{\gamma}(\tau))$:\\
    $\begin{aligned}
            |\var (\hat{\gamma}(\tau))|
            &\leq \frac{1}{N} \sum_{k = -(N - \tau - 1)}^{N - \tau - 1} \left| 1 - \frac{\tau + |k|}{N}\right| |\gamma^2 (n) + \gamma (n - \tau) \gamma (n + \tau)|\\
            &\leq \frac{1}{N} \sum_{k = -(N - \tau - 1)}^{N - \tau - 1} |\gamma^2 (n) + \gamma (n - \tau) \gamma (n + \tau)|\\
            &\leq \frac{2}{N} \sum_{k \in \mathbb{Z}} \gamma^2 (k) \xrightarrow[N \to \infty]{} 0 
    \end{aligned}$

    Applying the Bienaymé-Tchebychev inequality to $\hat{\gamma}$ with $\epsilon > 0$:
    $$\mathbb{P}[|\hat{\gamma}(\tau) - \mathbb{E}[\hat{\gamma}(\tau)]| > \epsilon] \leq \frac{\var (\hat{\gamma}(\tau))}{\epsilon}$$
    Since we have proven that the variance tends to 0, we conclude that $\hat{\gamma}(\tau)$ is consistent.
\end{solution}
\vspace{1cm}




Contrary to the correlogram, the periodogram is not consistent.
It is one of the most well-known estimators that is asymptotically unbiased but not consistent.
In the following question, this is proven for Gaussian white noise, but this holds for more general stationary processes.
%%% QUESTION 8 %%%
\begin{exercise}
    Assume that $X$ is a Gaussian white noise (variance $\sigma^2$) and let $A(f):=\sum_{n=0}^{N-1} X_n \cos(-2\pi f n/f_s$ and $B(f):=\sum_{n=0}^{N-1} X_n \sin(-2\pi f n/f_s$.
    Observe that $J(f) = (1/N) (A(f) + \iu B(f))$.
    \begin{itemize}
        \item Derive the mean and variance of $A(f)$ and $B(f)$ for $f=f_0, f_1,\dots, f_{N/2}$ where $f_k=f_s k/N$.
        \item What is the distribution of the periodogram values $|J(f_0)|^2$, $|J(f_1)|^2$, \dots, $|J(f_{N/2})|^2$.
        \item What is the variance of the $|J(f_k)|^2$? Conclude that the periodogram is not consistent.
        \item Explain the erratic behavior of the periodogram in Question~\ref{ex:wn-exp} by looking at the covariance between the $|J(f_k)|^2$.
    \end{itemize}
    
\end{exercise}

\begin{solution}
\begin{itemize}

%%% MEAN AND VARIANCE %%%
\item Since $\{X_n\}$ is a Gaussian white noise, we have $\mathbb{E}[X_n] = 0$ and $\mathrm{Var}(X_n) = \sigma^2$.  
Let
\[
A(f_k) = \sum_{n=0}^{N-1} X_n \cos\!\left(-\frac{2\pi k n}{N}\right),
\quad
B(f_k) = \sum_{n=0}^{N-1} X_n \sin\!\left(-\frac{2\pi k n}{N}\right).
\]

The mean of \(A(f_k)\) is
\[
\mathbb{E}[A(f_k)] = \sum_{n=0}^{N-1} \mathbb{E}[X_n] \cos\!\left(-\frac{2\pi k n}{N}\right) = 0.
\]

For the variance,
\[
\begin{aligned}
\mathrm{Var}[A(f_k)]
&= \sum_{n=0}^{N-1} \mathrm{Var}(X_n) \cos^2\!\left(\frac{2\pi k n}{N}\right)\\
&= \sigma^2 \sum_{n=0}^{N-1} \cos^2\!\left(\frac{2\pi k n}{N}\right)\\
&= \frac{\sigma^2}{2} \!\left(N + \Re\!\left(\sum_{n=0}^{N-1} e^{4i\pi k n/N}\right)\right)
= \frac{N\sigma^2}{2}.
\end{aligned}
\]

Similarly,
\[
\mathbb{E}[B(f_k)] = 0, \qquad
\mathrm{Var}[B(f_k)] = \frac{N\sigma^2}{2}.
\]

%%% DISTRIBUTION OF THE PERIODOGRAM VALUES %%%
\item The covariance between \(A(f_k)\) and \(B(f_k)\) is
\[
\begin{aligned}
\mathbb{E}[A(f_k)B(f_k)]
&= \mathbb{E}\!\left[\sum_{n,m} X_n X_m 
\cos\!\left(\frac{2\pi k n}{N}\right)
\sin\!\left(\frac{2\pi k m}{N}\right)\right]\\
&= \sum_{n=0}^{N-1} \mathbb{E}[X_n^2] 
\cos\!\left(\frac{2\pi k n}{N}\right)
\sin\!\left(\frac{2\pi k n}{N}\right)\\
&= \frac{\sigma^2}{2} \sum_{n=0}^{N-1} \sin\!\left(\frac{4\pi k n}{N}\right) = 0.
\end{aligned}
\]
Thus \(A(f_k)\) and \(B(f_k)\) are uncorrelated, and since they are jointly Gaussian, they are independent.

The periodogram value is
\[
|J(f_k)|^2 = \frac{1}{N}\big(A^2(f_k) + B^2(f_k)\big).
\]
Since \(A(f_k)\) and \(B(f_k)\) are independent \(\mathcal{N}(0, N\sigma^2/2)\) variables,
\[
\frac{2|J(f_k)|^2}{\sigma^2} \sim \chi^2_2.
\]
Hence,
\[
|J(f_k)|^2 \sim \sigma^2 \, \mathrm{Exp}(1),
\]
for \(0 < k < N/2\), while for \(k=0\) and \(k=N/2\), \(B(f_k)=0\) and we have
\[
|J(f_k)|^2 \sim \frac{\sigma^2}{2}\chi^2_1.
\]

%%% VARIANCE OF THE PERIODOGRAM %%%
\item From the exponential distribution,
\[
\mathbb{E}[|J(f_k)|^2] = \sigma^2, 
\qquad
\mathrm{Var}(|J(f_k)|^2) = \sigma^4.
\]
The variance does not depend on \(N\). Thus, the periodogram \( |J(f_k)|^2 \) is not a consistent estimator of the true spectral density since its variance does not vanish as \(N \to \infty\).

%%% ERRATIC BEHAVIOR %%%
\item The erratic behavior of the periodogram observed in Question~\ref{ex:wn-exp} arises precisely from this non-vanishing variance.
\end{itemize}
\end{solution}


\begin{exercise}\label{q:barlett}
    As seen in the previous question, the problem with the periodogram is the fact that its variance does not decrease with the sample size.
    A simple procedure to obtain a consistent estimate is to divide the signal into $K$ sections of equal durations, compute a periodogram on each section, and average them.
    Provided the sections are independent, this has the effect of dividing the variance by $K$. 
    This procedure is known as Bartlett's procedure.
    \begin{itemize}
        \item Rerun the experiment of Question~\ref{ex:wn-exp}, but replace the periodogram by Barlett's estimate (set $K=5$). What do you observe?
    \end{itemize}
    Add your plots to Figure~\ref{fig:barlett}.
\end{exercise}

\begin{solution}
    
\begin{figure}
    \centering
    \begin{minipage}[t]{0.3\textwidth}
    \centerline{\includegraphics[width=\textwidth]{example-image-golden}}
    \centerline{Periodogram ($N=200$)}
    \end{minipage}
    \begin{minipage}[t]{0.3\textwidth}
    \centerline{\includegraphics[width=\textwidth]{example-image-golden}}
    \centerline{Periodogram ($N=500$)}
    \end{minipage}
    \begin{minipage}[t]{0.3\textwidth}
    \centerline{\includegraphics[width=\textwidth]{example-image-golden}}
    \centerline{Periodogram ($N=1000$)}
    \end{minipage}
    \vskip1em
    \caption{Barlett's periodograms of a Gaussian white noise (see Question~\ref{q:barlett}).}
    \label{fig:barlett}
\end{figure}

\end{solution}
\section{Data study}

\subsection{General information}

\paragraph{Context.}
The study of human gait is a central problem in medical research with far-reaching consequences in the public health domain. This complex mechanism can be altered by a wide range of pathologies (such as Parkinson's disease, arthritis, stroke,\ldots), often resulting in a significant loss of autonomy and an increased risk of falls. Understanding the influence of such medical disorders on a subject's gait would greatly facilitate early detection and prevention of those possibly harmful situations. To address these issues, clinical and bio-mechanical researchers have worked to objectively quantify gait characteristics.

Among the gait features that have proved their relevance in a medical context, several are linked to the notion of step (step duration, variation in step length, etc.), which can be seen as the core atom of the locomotion process. Many algorithms have, therefore, been developed to automatically (or semi-automatically) detect gait events (such as heel-strikes, heel-off, etc.) from accelerometer and gyrometer signals.

\paragraph{Data.}
Data are described in the associated notebook.

\subsection{Step classification with the dynamic time warping (DTW) distance}

\paragraph{Task.} The objective is to classify footsteps and then walk signals between healthy and non-healthy.

\paragraph{Performance metric.} The performance of this binary classification task is measured by the F-score.


\begin{exercise}
Combine the DTW and a k-neighbors classifier to classify each step. Find the optimal number of neighbors with 5-fold cross-validation and report the optimal number of neighbors and the associated F-score. Comment briefly.
\end{exercise}

\begin{solution}

\end{solution}

\newpage
\begin{exercise}\label{q:class-errors}
Display on Figure~\ref{fig:class-errors} a badly classified step from each class (healthy/non-healthy).
\end{exercise}

\begin{solution}
\begin{figure}
    \centering
    \begin{minipage}[t]{\textwidth}
    \centerline{\includegraphics[width=0.6\textwidth]{example-image-golden}}
    \centerline{Badly classified healthy step}
    \end{minipage}
    \vskip1em
    \begin{minipage}[t]{\textwidth}
    \centerline{\includegraphics[width=0.6\textwidth]{example-image-golden}}
    \centerline{Badly classified non-healthy step}
    \end{minipage}
    \vskip1em
    \caption{Examples of badly classified steps (see Question~\ref{q:class-errors}).}
    \label{fig:class-errors}
\end{figure}
\end{solution}


\end{document}
