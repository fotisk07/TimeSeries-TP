\section{Node frequency classification}

For each node $v$, we use
\[
[W_f(1,v),\dots,W_f(M,v)]
\]
as features.

Nodes are classified as:
low (scales 1--3), medium (4--6), high (7--9) frequency.

\begin{exercise}
Apply this classification to:
least smooth signal, smoothest signal, first timestamp.
Display results on the map.
\end{exercise}

\begin{solution}
\begin{figure}
\centering
\begin{minipage}[t]{0.45\textwidth}
\includegraphics[width=\textwidth]{example-image-golden}
\centerline{(a) Least smooth}
\end{minipage}\hfill
\begin{minipage}[t]{0.45\textwidth}
\includegraphics[width=\textwidth]{example-image-golden}
\centerline{(b) Smoothest}
\end{minipage}

\vskip1em

\begin{minipage}[t]{0.45\textwidth}
\includegraphics[width=\textwidth]{example-image-golden}
\centerline{(c) First timestamp}
\end{minipage}
\caption{Node frequency classification}
\label{fig:node-classif}
\end{figure}
\end{solution}

\begin{exercise}
Display average temperature with marker colours equal to the majority class.
\end{exercise}

\begin{solution}
\begin{figure}
\centering
\includegraphics[width=\textwidth]{example-image-golden}
\caption{Average temperature with majority frequency class}
\end{figure}
\end{solution}
